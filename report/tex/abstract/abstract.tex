\doublespacing
\setlength{\parindent}{1cm}

\begin{center}
  \textbf{Abstract}
\end{center}

This senior thesis deals with the raga detection associated with Hindustani classical music. Indian classical music takes various forms depending on what part of India the dynamics behind it comes from. As advances in machine learning and deep learning have gone on to explore the classification of musical elements in the Western musicological framework, Indian classical music has only recently started to be explored using traditional, supervised machine learning methods through the classification of various ragas. Since not enough emphasis has been given on deep learning approaches for the classification of Indian classical music, I take the approach of designing a convolutional neural network (CNN) that can identify various ragas by the conversion of audio files to images that demonstrate spectral information as well as information related to pitch throughout the duration of the audio files associated with ragas. My contribution is that the CNN has shown competitiveness and potential through its high validation accuracy and I also provide a stepping stone towards the predictive power of many ragas during further experimentation and in the realtime detection of ragas through mobile app interfaces similar to the likes of Shazam.  
