\doublespacing
\setlength{\parindent}{1cm}

Scholars and researchers have used different mechanisms to identify ragas. In a system called Tansen, Pandey [ref] created a system in which a raga is automatically identified based on Hidden markov model. Pendekar et. al., [ref] were able to identify a raga by segmentation of audio signal via spectral flux and thereby identifying raga by using its pitch frequency. Certain other researchers such as Chelpa [ref] proposed a fuzzy set theory for generation of alap patterns.  In [ref], Sreedhar et. al., created a database of ragas and used the scale of the raga performance as a similarity metric using nearest neighbors algorithms. Within a scale, notes are matched with the existing sets of notes in the ragas in the database. The closest raga in the database is given as output for the test raga.
\par
Tzanetakis [ref] has also proposed various schemes in the English music classification based on their moods and styles of the performer as well as songs genre classification. Clustering is suggested as the classifier [ref]. Sentiment analysis of movie review based on naïve Bayes and genetic algorithm is suggested in [ref]. Since this methodology depends on the likelihood it can be connected to a wide assortment of spaces and results can be utilized as a part of numerous ways [ref]. Shetty et. al., [ref] has identified raga based upon arohana-avorahana pattern on different ragas using neural network technique.
\par
In a 2015 paper [ref], Sharma and Bali compared several ML classifiers to dataset of music labeled by 4 ragas: Des, Bhupali, Yaman and Todi. The audio performances are converted into .wav extension and chroma features are extracted using MIR toolbox in Matlab. A hop factor of 0.025 second is selected which gives 4719 frames. Then they were able to extract the Vadi swara, also known as the king swara whose magnitude and pitch are relatively greater than notes (swaras). Then they used the WEKA tool which includes a comprehensive collection of machine learning algorithms. The dataset of different ragas is classified using machine learning classifiers in WEKA. Classifiers they used are Random Forest, C4.5, Bayesian network and K-star. After performing comparison of classifiers on ragas, they observed that K-star gives the largest accuracy of 93.38\%, on dataset of ragas followed by the random forest with 92.64\%.  
