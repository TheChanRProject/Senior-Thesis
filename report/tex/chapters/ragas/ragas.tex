\doublespacing
\setlength{\parindent}{1cm}

\par

Raga can be identified by various parameters. The particular choice of notes, Ascending and Descending sequences (known as arohana and avarohana pattern), nature of inflexion on different notes (gamaka/meend), characteristic phrases (pakad) all can be helpful to classify a raga. These are further described below:

\begin{enumerate}
  \item Choice of Notes
  A rāga has a given set of notes (swaras), on a scale, ordered in melodies with musical motifs. The Indian tradition suggests a certain sequencing of how the musician moves from note to note for each rāga, in order for the performance to create a rasa (mood, atmosphere, essence, inner feeling) that is unique to each rāga. Theoretically, thousands of rāga are possible given 5 or more notes, but in practical use, the classical tradition has refined and typically relies on several hundred. For most artists, their basic perfected repertoire has some forty to fifty rāgas [8 -10]. Each raga has a different set of swaras that constitutes it. There must be the notes of the rag.  They are the allowd swar.  This concept is similar to the Western solfege. There must also be a modal structure.  This is called that in North Indian music and mela in Carnatic music. There is also the jati.  Jati is the number of notes used in the rag. Rāga in Indian classic music is intimately related to tala or guidance about "division of time", with each unit called a matra (beat, and duration between beats) [11]. A rāga is not a tune, because the same rāga can yield an infinite number of tunes [12].  A rāga is not a scale, because many rāgas can be based on the same scale. Each raga tends to have  a “Vadi” swara, a king swara on which maximum focus is given in a performance [1].  It is also known as the most frequently occurring swara in a particular raga. It is followed by Samvadi (next in importance), then Anuvadi.  The swaras that are not allowed in a particular raga are known as Vivadi swaras (enemy notes).
  \par
  \item Arohana/Avarohana
  There must also be the ascending and descending sequence of notes.  This is called arohana /avarohana. Arohana and avarohana are the descriptions of how the raga moves.  The arohana, also called aroh or arohi, is the pattern in which a raga ascends the scale.  The avarohana, also called avaroh or avarohi, describes the way that the raga descends the scale.  Both the arohana and avarohana may use certain characteristic twists and turns.
  \par
  \item Pakad
  The pakad or swarup, is a defining phrase or a characteristic pattern for a raga.  This is often a particular way in which a raga moves; for instance the "Pa M'a Ga Ma Ga" is a tell-tale sign for Raga Bihag, or "Ni Re Ga M'a" is a telltale sign for Yaman.  Often the pakad is a natural consequence of the notes of arohana / avarohana (ascending and descending structures).  However, sometimes the pakad is unique and not implied by the notes of the arohana /avarohana.  It is customary to enfold the pakad into the arohana / avarohana to make the ascending and descending structures more descriptive.
  \par
  \item Gamakas
  Gamakas are better known as ornamentations used in Hindustani music system. These are inflexions and rapid oscillatory movements taken across swaras.
\end{enumerate}
\par

We now take one common raga as a running example and explain how the notes behave with respect to the above definitions and terms. Yaman emerged from the parent musical scale of Kalyan. Considered to be one of the most fundamental ragas in Hindustani tradition, it is thus often one of the first ragas taught to students. 
