\doublespacing
\setlength{\parindent}{1cm}

\par

I have been exposed to Hindustani classical music ever since I was small. I was raised in an Indian household to immigrant parents and my mom is a vocalist and an exponent of Indian classical music. Indian classical music is categorized into two distinct forms: Hindustani and Carnatic, which are practiced in North and Southern India. Unlike western classical music, Indian classical music is very old form and typically doesn’t have clear structures but largely depends on the performers or instrument players own elaboration of a melody. Indian classical music is defined by two basic elements – it must follow a Raga (classical mode), and a specific rhythm, the Taal [1]. Most compositions follow a Raga and I have noticed that even experts sometimes have difficulty of telling which Raga a particular song or composition is based on. This is particularly challenging for novices or beginners. Being a data science major, I quickly became attracted to this problem of “Raga detection”. My intuition said that machine learning algorithms and techniques could help classify a composition into a main Raga on which it is based. Thus begins my journey to explore and hence this senior thesis. \par

I will mainly focus on North Indian form which is referred to as the Hindustani classical music. Compositions in Hindustani classical music also are based on a drone, i.e., a continual pitch that sounds throughout the concert, which is tonic [2]. This drone acts as a point of reference as the performer is expected to come back to this home base after a flight of improvisation. The variations and complexity in Hindustani music stems from its use of notes that comprise a Raga. There are seven main musical notes (also called swaras) – Sa, Re, Ga, Ma, Pa, Dha and Ni – along with five intermediate notes (flats and sharps) referred to as “vikrit swaras”. The seven notes are referred to as Shuddha and belongs to the saptak (a scale). The flat notes are called “komal” and the sharp notes are called “teevra”. A raga consists of at least five notes, and each raga provides the musician with a musical framework within which to improvise [3, 4 5]. The specific notes within a raga can be reordered and improvised by the musician. Ragas range from small ragas like Bahar and Shahana that are not much more than songs to big ragas like Malkauns, Darbari and Yaman, which have great scope for improvisation and for which performances can last over an hour. Each raga traditionally has an emotional significance and symbolic associations such as with season, time and mood [6].  The raga is considered a means in Indian musical tradition to evoke certain feelings in an audience. Hundreds of raga are recognized in the classical tradition, of which about 30 are common [7]. \par

The swaras in a raga can be played in three octaves, the first or lower octave starting from 130 Hz, then middle octave starting at 260 Hz; and upper octave from 520 Hz. The artists are allowed to improvise over the definitions of raga to create their own renditions. If you listen to two performance of the same raga, they may sound strikingly different to novice ears, though they still retain the rules and defining qualities of ragas. \par

The rest of the thesis is organized as follows. In Chapter 2, we take a closer look at ragas to understand certain nuances and patterns they exhibit. In Chapter 3, we discuss Librosa, a python package for audio and music signal processing. In Chapter 4, we cover background and related work done on identifying Indian ragas using machine learning and other methods. In Chapter 5, we present a deep learning methodology for raga classification using Convolutional Neural Networks (CNN). Chapter 6 presents the details of the dataset and image data generation. In Chapter 7, I present the data preprocessing steps for the CNN algorithm to be used for raga detection. In Chapter 8 I present the results and analysis of this project. Finally in conclude in Chapter 9 with a summary of our findings and possible future work. 
